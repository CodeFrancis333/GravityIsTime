\chapter{How the theory passes the smell test}
\label{chap:smell-test}

\begin{table}[h]
\centering
\begin{tabular}{cccc}
\hline
\textbf{Test} &
\textbf{GR value} &
\textbf{$\tau$-model} &
\textbf{Pass}\\
\hline
Mercury perihelion shift & $43''/\text{century}$ & $43''$ & \checkmark\\
Solar light bending      & $1.75''$              & $1.75''$ $(\alpha\approx2)$ & \checkmark\\
Shapiro delay (Earth--Mars) & $240\,\mu\text{s}$ & $240\,\mu\text{s}$ & \checkmark\\
Pound–Rebka red-shift    & $7\times10^{-15}$     & $7\times10^{-15}$ (same) & \checkmark\\
\end{tabular}
\caption{Classical tests matched by the $\tau$-model.}
\end{table}

\begin{equation}
\Delta\theta=\frac{4GM}{c^{2}b} \tag{Eq.\,2}  
\end{equation}
\noindent
(single-line proof in Appendix B).

\section{A Minimal potential that survives every test}

\[
V(\tau)=\tfrac12 m_\tau^{2}\tau^{2}+\lambda\tau^{4}, 
\qquad
m_\tau = 10^{-28}\,\text{eV}, 
\quad
\lambda = 10^{-4}.
\]

Yukawa range $\lambda_\tau\approx200\;\text{kpc}$ shapes cluster scales yet is $<10^{-11}$ at 1 AU.  
The quartic term stabilizes collision-less $\tau$-halos (see Section 4.3).
