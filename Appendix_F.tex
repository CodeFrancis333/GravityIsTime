\chapter{Python Scripts Used to Generate Figures}

Each file below reproduces the figure listed in its caption without any
external dependencies beyond \texttt{numpy}, \texttt{scipy}, and
\texttt{matplotlib}.  All scripts are also available in the online
repository at \url{https://github.com/CodeFrancis333/GravityIsTime.git} (tag
\texttt{v1.0}).

\bigskip
\noindent
% ----------------------------------------------------------------
\lstinputlisting[style=pyStyle,
                 caption={\texttt{fig01\_clock\_gradient.py} —
                          generates Fig.\,1 (Clock-field gradient).},
                 label={lst:clockGradient}]
{code/fig01_clock_gradient.py}

% ----------------------------------------------------------------
\lstinputlisting[style=pyStyle,
                 caption={\texttt{fig02\_grav\_redshift.py} —
                          generates Fig.\,2 (Gravitational red-shift).},
                 label={lst:gravRedshift}]
{code/fig02_grav_redshift.py}

% ----------------------------------------------------------------
\lstinputlisting[style=pyStyle,
                 caption={\texttt{fig03\_dist\_modulus.py} —
                          generates Fig.\,3 (Distance-modulus curves).},
                 label={lst:distModulus}]
{code/fig03_dist_modulus.py}

% ----------------------------------------------------------------
\lstinputlisting[style=pyStyle,
                 caption={\texttt{fig04\_background\_Ez.py} —
                          generates Fig.\,4 (Background expansion $E(z)$).},
                 label={lst:backgroundEz}]
{code/fig04_background_Ez.py}

% ----------------------------------------------------------------
\lstinputlisting[style=pyStyle,
                 caption={\texttt{fig05\_omega\_cross.py} —
                          generates Fig.\,5 (Matter vs.\ $\tau$ density).},
                 label={lst:omegaCross}]
{code/fig05_omega_cross.py}

% ----------------------------------------------------------------
\lstinputlisting[style=pyStyle,
                 caption={\texttt{fig06\_bao\_residuals.py} —
                          generates Fig.\,6 (BAO residuals).},
                 label={lst:baoResiduals}]
{code/fig06_bao_residuals.py}

% ----------------------------------------------------------------
\lstinputlisting[style=pyStyle,
                 caption={\texttt{fig07\_bao\_cmb.py} —
                          generates Fig.\,7 (BAO + CMB expansion band).},
                 label={lst:baoCmb}]
{code/fig07_bao_cmb.py}

% ----------------------------------------------------------------
\lstinputlisting[style=pyStyle,
                 caption={\texttt{fig08\_redshift\_drift.py} —
                          generates Fig.\,8 (Red-shift drift).},
                 label={lst:redshiftDrift}]
{code/fig08_redshift_drift.py}

% ----------------------------------------------------------------
\lstinputlisting[style=pyStyle,
                 caption={\texttt{fig09\_Ez\_curve.py} —
                          generates Fig.\,9 (Alternative $E(z)$ curve).},
                 label={lst:EzCurve}]
{code/fig09_Ez_curve.py}

% ----------------------------------------------------------------
\lstinputlisting[style=pyStyle,
                 caption={\texttt{fig10\_weak\_lensing.py} —
                          generates Fig.\,10 (Weak-lensing residual).},
                 label={lst:weakLensing}]
{code/fig10_weak_lensing.py}

% ----------------------------------------------------------------
\lstinputlisting[style=pyStyle,
                 caption={\texttt{fig11\_Twobody\_Cluster.py} —
                          generates Fig.\,11 (Two-body cluster simulation).},
                 label={lst:twoBodyCluster}]
{code/fig11_Twobody_Cluster.py}

% ----------------------------------------------------------------
\lstinputlisting[style=pyStyle,
                 caption={\texttt{Wave\_Packet.py} —
                          simulation underlying Chapter 5 wave-packet plot.},
                 label={lst:wavePacket}]
{code/Wave_Packet.py}

\bigskip
\noindent
All scripts were executed with \texttt{Python 3.11}.
