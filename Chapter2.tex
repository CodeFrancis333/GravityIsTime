\chapter{The Core Idea}
\label{chap:core-idea}

The Universe is filled with a scalar \emph{clock field} $\tau(x)$.  
Where mass or energy piles up, local clocks slow; where space empties out, they speed up. Gradients of this field pull matter that pull is gravity. Near a black hole the clock almost stops, red-shifting escaping light. In other words, every point in the Universe carries a “local clock” $\tau(x)$.  
Mass or energy slows the clock; emptiness makes it run fast. Gradients in that beat pull matter what we call gravity.\\
\\
The field obeys a curved-space analogue of Poisson’s law in which \emph{time} replaces Newton’s potential:

\begin{equation}
\Box\!\bigl(\Box \tau\bigr)=\kappa\,T,
\qquad
\kappa=\frac{8\pi G}{c^{4}},
\label{eq:master}
\end{equation}

\noindent
where $\Box$ is the covariant d’Alembertian and $T \equiv T^{\mu}{}_{\mu}$ is the trace of the stress–energy tensor.  
In the weak-field limit $\Box \to \nabla^{2}$ and $\tau$ plays the role of the Newtonian potential $\Phi$.  
Readers in a hurry may read \eqref{eq:master} heuristically as “energy bends the flow of time.”  
Appendix~A presents the two-line derivation and verifies covariant energy–momentum conservation.
