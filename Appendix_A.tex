\chapter{Deriving the Field Equation \texorpdfstring{$\bigl(\Box\tau=\kappa T\bigr)$}{(□τ = κT)}}

Throughout this thesis we adopt the \(\bigl[-,+,+,+\bigr]\) metric
signature.  Greek indices run over \(0\dots3\) and
\(\nabla_{\!\mu}\) denotes the Levi-Civita connection of the
background metric~\(g_{\mu\nu}\).

%-----------------------------------------------------------------
\section*{A.1 \, Action}

We begin from the diffeomorphism-invariant action for a minimally
coupled scalar \emph{clock field}~\(\tau\):
\begin{equation}
  S \;=\;
  \int\! d^{4}x\,\sqrt{-g}\;
     \bigl[\,
       \tfrac12\,g^{\mu\nu}\partial_{\mu}\tau\partial_{\nu}\tau
       - V(\tau)\;-\;\tau\,T
     \bigr],
  \tag{A-1}\label{A1}
\end{equation}
with \(T = T^{\mu}{}_{\mu}\) the trace of the
ordinary matter stress–energy tensor.  The coupling constant hidden in
front of \(\tau T\) will be identified below.

%-----------------------------------------------------------------
\section*{A.2 \, Variation w.r.t.\ $\tau$}

\paragraph{1. Vary \(\tau\) (metric fixed).}
\begin{equation}
  \delta S
  = \int d^{4}x\,\sqrt{-g}\;
      \Bigl[
         g^{\mu\nu}(\partial_{\mu}\tau)(\partial_{\nu}\delta\tau)
         - V'(\tau)\,\delta\tau
         - T\,\delta\tau
      \Bigr].
  \tag{A-2}
\end{equation}

\paragraph{2. Integrate by parts.}
Using \(\nabla_{\!\mu}(\sqrt{-g}\,X^{\mu})=
  \partial_{\mu}(\sqrt{-g}\,X^{\mu})\) and discarding the boundary
term for \(\delta\tau\!=\!0\) at infinity,
\begin{equation}
  \int d^{4}x\,\sqrt{-g}\,g^{\mu\nu}(\partial_{\mu}\tau)
                                  (\partial_{\nu}\delta\tau)
  = - \int d^{4}x\,\sqrt{-g}\,\Box\tau\;\delta\tau,
  \tag{A-3}
\end{equation}
where \(\Box \equiv g^{\mu\nu}\nabla_{\!\mu}\nabla_{\!\nu}\) is the
covariant d’Alembertian.

\paragraph{3. Euler–Lagrange equation.}
Because \(\delta\tau\) is arbitrary,
\begin{equation}
  \boxed{\,
  \Box\tau - V'(\tau) = \kappa\,T
  \;} \quad
  \text{with}\quad
  \kappa \equiv \frac{8\pi G}{c^{4}}.
  \tag{A-4}\label{A4}
\end{equation}

\paragraph{4. Minimal model.}
Setting \(V'=0\) gives the field equation quoted in the main text:
\begin{equation}
  \boxed{\;
    \Box\tau=\kappa T
  \;} .
  \tag{A-5}\label{A5}
\end{equation}

%-----------------------------------------------------------------
\section*{A.3 \, Stress–energy of the clock field}

Variation of \eqref{A1} with respect to \(g^{\mu\nu}\) (keeping
\(\tau\) fixed) yields
\[
  T^{\mu\nu}_{\tau}
  = \partial^{\mu}\tau\,\partial^{\nu}\tau
   \;-\; \tfrac12 g^{\mu\nu}(\partial\tau)^{2}
   \;-\; g^{\mu\nu}V(\tau).
\]
This enters gravitational lensing and cluster dynamics exactly as used
in Chapters 4–6 even though the background metric itself is
non-dynamical in the minimal theory.

%-----------------------------------------------------------------
\section*{A.4 \, Energy–momentum conservation (“action lock”)}

Because the action \eqref{A1} is diffeomorphism-invariant, a metric
variation plus Noether’s theorem gives
\begin{equation}
  \nabla_{\!\mu}\bigl(T^{\mu\nu}_{\text{matter}}
                     +T^{\mu\nu}_{\tau}\bigr)=0,
  \tag{A-6}\label{A6}
\end{equation}
which guarantees total energy–momentum conservation.

%-----------------------------------------------------------------
\section*{A.5 \, From single to double box}

Taking a further covariant d’Alembertian of \eqref{A4} and using
\(\nabla_{\!\mu}T^{\mu\nu}=0\) one finds
\[
   \boxed{\;
     \Box\bigl(\Box\tau\bigr)=\kappa\,T
   \;},
\]
the curved-space analogue of Poisson’s equation (Eq.\,1 in the main
chapters).

\paragraph{Dimensional check.}
The clock field carries units of time \([\tau]=\mathrm{s}\); with
\(\kappa=8\pi G/c^{4}\) the product \(\kappa T\) has the same dimension
as \(\Box\tau\), confirming consistency.

%-----------------------------------------------------------------
\subsection*{Boundary condition}

The surface term in Eq.\,(A-3) vanishes for
\(\delta\tau\to0\) as \(r\to\infty\); the same condition ensures a
well-posed variational problem for all asymptotically-flat solutions
discussed in this thesis.
