\chapter{Limitations and Future Tests}

Despite its economy, the scalar–time program can be ruled out––or strongly
constrained––by several near-term observations.

\begin{itemize}
  \item \textbf{Scalar GW mode.}  During LIGO–Virgo O5 the minimal clock-field
        model predicts a single \emph{breathing} polarization and no tensor
        modes.  Current bounds already require the scalar energy fraction to be
        $\le 10\,\%$ of the tensor total at \SI{100}{Hz}.  O5 improves that by
        roughly an order of magnitude.  A $\ge1\%$ breathing amplitude would be
        a spectacular confirmation; a null result at that level would either
        falsify the minimal theory or demand a short-range screening mechanism
        at detector scales.

  \item \textbf{Cluster sample.}  The Bullet-Cluster stress test succeeds
        because collision-less $\tau$-halos reproduce the observed
        \SI{200}{kpc} mass–gas offset.  DESI, \textit{Euclid}, and
        \textit{Rubin} are expected to discover $\gtrsim20$ high-velocity,
        post-merger clusters within five years.  The model predicts each should
        show a comparable offset set by the Yukawa range
        $\lambda_{\tau}\!\approx\!\SI{200}{kpc}$.  A population with \emph{no}
        offset—or with offsets far exceeding $\lambda_{\tau}$—would challenge
        the halo prescription.

  \item \textbf{CMB peaks.}  A full Boltzmann run (now in progress) must keep
        the 2nd-to-1st acoustic-peak ratio within the \textit{Planck} error
        bars.  Preliminary runs including $\tau$-field perturbations indicate
        the ratio can be held within $\sim5\%$ of \(\Lambda\)CDM while
        retaining the improved low-$z$ expansion fit.  Failure would require
        additional (perhaps vector) degrees of freedom or a refined potential.
\end{itemize}

\begin{table}[htbp]
  \centering
  \caption{Near-term falsification targets for the scalar–time model.}
  \label{tab:Falsifiers}
  \renewcommand{\arraystretch}{1.2}
  \begin{tabular}{@{} l c c c @{}}
    \toprule
    \textbf{Test} & \textbf{Observable} & \textbf{Instrument} & \textbf{Timeline} \\
    \midrule
    Twin optical clocks & $\Delta\nu/\nu \,\approx\, 2\times10^{-18}$ & MIGA & 2026 \\
    \hline
    Josephson shift & $\Delta I_c/I_c \,\approx\, 4\times10^{-15}$ & Cryo-SQUID & Lab-ready \\
    \hline
    Scalar GW energy & $< 1\%$ of tensor & LIGO O5 & 2025–26 \\
    \hline
    Cluster offsets & $\ge 100$–\SI{200}{kpc} & DESI lens maps & 2027 \\
    \bottomrule
  \end{tabular}
\end{table}

